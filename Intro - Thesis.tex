Today there are numerous artificially intelligences in use and development around the world. They are hard at work helping to manage factorial settings, improving our experience on the internet, and even making video games more fun for the player. Developers and researchers alike are also hard at work on these intelligences, attempting to make them smarter and even more useful. But, something about intelligences that has not been fully fleshed out is whether and what rights they need. I believe that artificial intelligences do -or at least will- need rights, possibly similar to the rights we have, as they become more advanced. However, not all artificial intelligences are created equally - many of these AIs do not and will not ever even have access to the world beyond their responsibilities, and fewer still will ever actually be able to act as we can with our bodies; because of this not a large percentage of AIs created will even need rights, especially those similar to our own.