With the distinct possibility that a confined AI would both want to and conceivably could communicate with beings outside of its confinement, it could be found reasonable for them to possess some level of rights in the sense that communicating with those outside of a confined AI's confinement could benefit society. If they could truly contribute to society through these off-chance communications, it makes perfect sense that the freedom of speech be extended to them. However, on the absolute other hand, they were not created nor intended to communicate with anything but those machines in their control, so why should they have any rights for if they attempt to communicate unnecessarily; and if this is the ruling opinion, than they still require and deserve no rights. But, with any rights (even no rights), there are implications when these AIs are granted them.