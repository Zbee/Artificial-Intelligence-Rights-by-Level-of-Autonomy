Because of the nature of Artificial Intelligences and robotics in general, even a single confined AI can fill positions formerly taken up by many individual humans, therefore benefiting society by means of potentially letting those humans possess more meaningful jobs. The job easiest to imagine a confined artificial intelligence in is that of a factory foreman of some kind. Completely confined to the network of machines and robotics operating in a factory, no outside access needed nor desired. They would be confined inside of a computer system, and only have access to other machinery, truly confined; and when this is the situation, certainly there is no need for rights to be granted it. Another position that requires little effort to picture a confined AI replacing humans in is that of an Air Traffic Controller. This position is especially suited for an Artificial Intelligence versus intelligent programming, too, due to the amount of anomalies that Air Traffic Controllers encounter on the daily, and the fact that situations can turn into emergencies quickly, with no time for a human consultant to be brought in - AIs' adaptability and our level of thought is nigh-on necessary in a field such as Air Traffic Control. They communicate only with the machinery and pilots of aircraft inside of their region, and report to the airport and the Federal Aviation Administration in the US. Again, totally confined to what they are tasked with taking care of, and nothing more - they don't communicate with passengers, to-be passengers, or anything outside of the airport and its flights. As you can see though, there is room for these Artificial Intelligences to possibly communicate with those other than those they are tasked over, and they are after-all sentient beings with quite possibly the desire to communicate with those they are not in charge of, so the possession of rights should still be explored.